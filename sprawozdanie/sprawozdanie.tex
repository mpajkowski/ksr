\documentclass{classrep}
\usepackage[utf8]{inputenc}
\usepackage{color}

\studycycle{Informatyka, studia zaoczne, I st.}
\coursesemester{VI}

\coursename{Komputerowe systemy rozpoznawania}
\courseyear{2018/2019}

\courseteacher{dr hab. inż. Adam Niewiadomski}
\coursegroup{niedziela, 12.00}

\author{
  \studentinfo{Marcin Pajkowski}{211968} \and
  \studentinfo{Rafał Warda}{214067}
}

\title{Zadanie 1: Ekstrakcja cech, miary podobieństwa, klasyfikacja}

\begin{document}
\maketitle

\section{Cel}
Celem zadania było zaprojektowanie i stworzenie szkieletu aplikacji do klasyfikacji metodą k-NN na
tyle uniwersalnego, aby był on niezależny od typu obiektów, które podlegają klasyfikacji.

\section{Wprowadzenie}
\subsection{Algorytm k-najbliższych sąsiadów (k-NN)}
Algorytm k-NN (\textit{k-nearest neightbours}) jest używany do klasyfikacji danych.
Element zbioru danych $ R $ jest reprezentowany jako wektor cech oraz etykieta - może być
ona słownym opisem jednoznacznie określającym przynależność obiektu $ x \in R $ do danej
klasy. Zbiór $R$ należy podzielić na dwa podzbiory - podzbiór uczący $ R_u $ $ M
$-elementowy ($M < N$) oraz podzbiór testowy $ R_t $ $ N-M $-elementowy.  \newline\newline
Następnie dla każdego elementu ze zbioru $ R_t $:

\begin{enumerate}
  \item Za pomocą metryki obliczamy dystans do każdego elementu ze zbioru $ R_u $.
  \item Spośród uzyskanych odległości wybieramy $k$ najmniejszych.
  \item Spośród $ k $ obiektów wybieramy najliczniej reprezentowaną etykietę.
  \item Przypisujemy tę etykietę do badanego obiektu ze zbioru $ R_t $.
\end{enumerate}

W następnym kroku należy posłużyć się etykietami elementów zbioru $ R_t $ aby sprawdzić trafność
prognoz - należy porównać je z etykietami przypisanymi przez algorytm.

\subsection{Ekstrakcja cech}
Celem ekstrakcji cech jest zamiana zbioru obiektów rzeczywistych na ich reprezentację -
otrzymujemy wektor cech wraz z wagami - są one reprezentacją badanego obiektu.
Taka reprezentacja jest wygodna dla maszynowego przetwarzania informacji oraz
umożliwia porównywanie obiektów ze sobą. W przypadku ekstrakcji cech z tekstu - cechami
mogą być słowa występujące we wszystkich tekstach.

Aby efektywnie pracować na tekstowych zbiorach danych należy spośród wszystkich słów wybrać te
istotne. Wiele słów w języku naturalnym pełni wyłącznie funkcje pomocnicze a ich obecność
jest powiązana z specyficznymi dla danego języka regułami gramatycznymi, syntaktycznymi
lub semantycznymi. Listę słów, które będziemy pomijać w procesie ekstrakcji cech
nazywamy \textit{stoplistą}.


Słowa języka naturalnego są odmieniane co może powodować pewne kłopoty podczas ekstrakcji
cech - słowa różniące się wyłącznie końcówką będą traktowane jako niezależne cechy. Aby
usprawnić ekstrakcję cech, a w konsekwencji przeprowadzić skuteczniejszą klasyfikację
należy wykonać proces \textit{stemizacji}. Polega on na wyodrębnieniu korzenia spośród
słów korpusu.

\subsubsection{Częstotliwość występowania termów}
Metoda ta sprawdza częstotliwość występowania danego termu w dokumencie

\begin{equation}
  tf_{i,j} = \frac{n_{i,j}}{\sum_k n_{k,j}}
\end{equation}

\subsubsection{TF-IDF}
Aby wyznaczyć wagi słów kluczowych można posłużyć się algorytmem TF-IDF (\textit{term
frequency - inverse document frequency}).

Wartość TF-IDF jest wyrażona wzorem
\begin{equation}
  (tf\mbox{-}idf)_{i,j} = tf_{i,j} \times idf_{i}
\end{equation}

$ tf{i,j} $ to wartość \textit{term frequency}, definiuje się ją jako

\begin{equation}
  tf_{i,j} = \frac{n_{i,j}}{\sum_k n_{k,j}}
\end{equation}

$ n_{i, j} $ jest liczbą wystąpień termu w danym tekście $ d_{j} $, a mianownik jest
liczbą wszystkich termów w $ d_{j} $.

$ idf_{i} $ to wartość \textit{inverse document frequency} opisana wzorem
\begin{equation}
  idf_{i} = \log \frac{|R|}{|\{r: t_i \in r\}|}
\end{equation}

gdzie licznik oznacza liczbę dokumentów ze zbioru $ R $ a mianownik liczbę dokumentów
zawierających przynajmniej jedno wystąpienie termu $ t_{i} $.

\subsubsection{Odległość słowa kluczowego od początku tekstu}
Jest to prosta metoda ekstrakcji cech, składają się na nią następujące czynności:

\begin{enumerate}
  %
  %\item Należy stworzyć listę wszystkich termów występujących w korpusie ($ T $)
  %  $$ T = R \rightarrow \{d_{i} | d_{i} \in R \} \rightarrow \{t_{n} | t_{n} \in d_{i}\} $$
  %
  \item Dla każdego dokumentu $ d_{i} $ w korpusie należy stworzyć listę słów ($ U $)
    $$ U = \{ u_{k} | u_{k} \in d_{i} \} $$
  \item Wagą każdego słowa $ u $ w dokumencie $ d_{j} $ jest $ k $ - czyli jego pozycja w
    dokumencie.
\end{enumerate}

\section{Opis implementacji}
{\color{blue}
Należy tu zamieścić krótki i zwięzły opis zaprojektowanych klas oraz powiązań
między nimi. Powinien się tu również znaleźć diagram UML  (diagram klas)
prezentujący najistotniejsze elementy stworzonej aplikacji. Należy także
podać, w jakim języku programowania została stworzona aplikacja. }

\section{Materiały i metody}
{\color{blue}
W tym miejscu należy opisać, jak przeprowadzone zostały wszystkie badania,
których wyniki i dyskusja zamieszczane są w dalszych sekcjach. Opis ten
powinien być na tyle dokładny, aby osoba czytająca go potrafiła wszystkie
przeprowadzone badania samodzielnie powtórzyć w celu zweryfikowania ich
poprawności (a zatem m.in. należy zamieścić tu opis architektury sieci,
wartości współczynników użytych w kolejnych eksperymentach, sposób
inicjalizacji wag, metodę uczenia itp. oraz informacje o danych, na których
prowadzone były badania). Przy opisie należy odwoływać się i stosować do
opisanych w sekcji drugiej wzorów i oznaczeń, a także w jasny sposób opisać
cel konkretnego testu. Najlepiej byłoby wyraźnie wyszczególnić (ponumerować)
poszczególne eksperymenty tak, aby łatwo było się do nich odwoływać dalej.}

\section{Wyniki}
{\color{blue}
W tej sekcji należy zaprezentować, dla każdego przeprowadzonego eksperymentu,
kompletny zestaw wyników w postaci tabel, wykresów itp. Powinny być one tak
ponazywane, aby było wiadomo, do czego się odnoszą. Wszystkie tabele i wykresy
należy oczywiście opisać (opisać co jest na osiach, w kolumnach itd.) stosując
się do przyjętych wcześniej oznaczeń. Nie należy tu komentować i interpretować
wyników, gdyż miejsce na to jest w kolejnej sekcji. Tu również dobrze jest
wprowadzić oznaczenia (tabel, wykresów) aby móc się do nich odwoływać
poniżej.}

\section{Dyskusja}
{\color{blue}
Sekcja ta powinna zawierać dokładną interpretację uzyskanych wyników
eksperymentów wraz ze szczegółowymi wnioskami z nich płynącymi. Najcenniejsze
są, rzecz jasna, wnioski o charakterze uniwersalnym, które mogą być istotne
przy innych, podobnych zadaniach. Należy również omówić i wyjaśnić wszystkie
napotakane problemy (jeśli takie były). Każdy wniosek powinien mieć poparcie
we wcześniej przeprowadzonych eksperymentach (odwołania do konkretnych
wyników). Jest to jedna z najważniejszych sekcji tego sprawozdania, gdyż
prezentuje poziom zrozumienia badanego problemu.}
\section{Wnioski}
{\color{blue}W tej, przedostatniej, sekcji należy zamieścić podsumowanie
najważniejszych wniosków z sekcji poprzedniej. Najlepiej jest je po prostu
wypunktować. Znów, tak jak poprzednio, najistotniejsze są wnioski o
charakterze uniwersalnym.}


\begin{thebibliography}{0}
\end{thebibliography}
{\color{blue}
Na końcu należy obowiązkowo podać cytowaną w sprawozdaniu
literaturę, z której grupa korzystała w trakcie prac nad zadaniem (przykład na
końcu szablonu)}
\end{document}
